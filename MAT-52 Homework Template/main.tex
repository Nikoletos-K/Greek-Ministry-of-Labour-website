\documentclass[12pt]{article}
\usepackage[LGR, T1]{fontenc}
\usepackage[greek]{babel}
\usepackage{natbib}
\usepackage{url}
\usepackage[utf8x]{inputenc}
\usepackage{graphicx}
\graphicspath{{../images/}}
\usepackage{parskip}
\usepackage{fancyhdr}
\usepackage{vmargin}
\setmarginsrb{3 cm}{2.5 cm}{3 cm}{2.5 cm}{1 cm}{1.5 cm}{1 cm}{1.5 cm}

\title{Εργασία 1}								% Title
\author{Επικοινωνία Ανθρώπου Μηχανής}								% Author
\date{\today}											% Date

\makeatletter
\let\thetitle\@title
\let\theauthor\@author
\let\thedate\@date
\makeatother

\pagestyle{fancy}
\fancyhf{}
\rhead{\theauthor}
\lhead{\thetitle}
\cfoot{\thepage}

\begin{document}

%%%%%%%%%%%%%%%%%%%%%%%%%%%%%%%%%%%%%%%%%%%%%%%%%%%%%%%%%%%%%%%%%%%%%%%%%%%%%%%%%%%%%%%%%

\begin{titlepage}
	\centering
    \vspace*{0.5 cm}
    % \includegraphics[scale = 0.75] {\textlatin{logo_el.png}} \\[1.0 cm]	% University Logo
    \textsc{\LARGE \newline\newline Εθνικό και Καποδιστριακό Πανεπιστήμιο Αθηνών}\\[2.0 cm]	% University Name
	\textsc{\Large ΥΣ08: Επικοινωνία Ανθρώπου Μηχανής}\\[0.5 cm]				% Course Code
	\rule{\linewidth}{0.2 mm} \\[0.4 cm]
	{ \huge \bfseries \thetitle}{\\
	\rule{\linewidth}{0.2 mm} \\[1.5 cm]
	
	\begin{minipage}{0.5\textwidth}
		\begin{flushleft} \large
			\emph{Καθηγητής:}\\
			Μαρία Ρούσοου\\
            Τμήμα Πληροφορικής και Τηλεπικοινωνιών\\
			\end{flushleft}
			\end{minipage}~
			\begin{minipage}{0.4\textwidth}
            
			\begin{flushright} \large
			\emph{Ομάδα:} \\
			Ιγγλέζου Μυρτώ\\
            Νικολέτος Κωνσταντίνος\\
		\end{flushright}
        
	\end{minipage}\\[2 cm]}
	
	
    \thedate

\end{titlepage}

%%%%%%%%%%%%%%%%%%%%%%%%%%%%%%%%%%%%%%%%%%%%%%%%%%%%%%%%%%%%%%%%%%%%%%%%%%%%%%%%%%%%%%%%%

\tableofcontents
\pagebreak

%%%%%%%%%%%%%%%%%%%%%%%%%%%%%%%%%%%%%%%%%%%%%%%%%%%%%%%%%%%%%%%%%%%%%%%%%%%%%%%%%%%%%%%%%

\section{Ερώτημα 1o}

\subsection{Οι χρήστες δεν διαβάζουν απλώς “σκανάρουν”}
Οι χρήστες καθώς κοιτούν μια ιστοσελίδα, ψάχνουν σημεία-λέξεις που θα τους οδηγήσουν στην γρήγορη κατανόηση του κειμένου χωρίς να χρειαστεί να διαβάσουν όλο το περιεχόμενο. Στην συγκεκριμένη ιστοσελίδα δεν παρατηρούμε τέτοια δομή, καθώς δεν τονίζονται λέξεις που πιθανόν να αναζητούν οι χρήστες, όπως π.χ. Ανακοινώσεις, \textlatin{Covid-19}, Εργασιακά, Ασφάλεια. Επιπλέον δεν υπάρχει συγκεκριμένη οργάνωση και διαχωρισμός των πληροφοριών, ενώ αντίθετα δίνεται η εντύπωση στον χρήστη πως απλά έχουν "πεταχτεί" πληροφορίες στην οθόνη, με αποτέλεσμα την σύγχησή του.

\subsection{Απλότητα}
Σπάνια οι χρήστες επισκέπτονται μια ιστοσελίδα για τον σχεδιασμό της, στις περισσότερες περιπτώσεις αυτό που τους ενδιαφέρει είναι το περιεχόμενο και η ευκολία πλοήγησης. Εδώ βλέπουμε έντονα και ποικίλα χρώματα σε πολλά σημεία της σελίδας, χαρακτηριστικό που μπορέι να αποσυντονίζει τον χρήστη. Επιπλέον οι οργανισμοί και τα προγράμματα είναι τοποθετημένα σε μια στήλη, σε πολύ κοντινές αποστάσεις, με μικρά και δυσανάγνωστα γράμματα και χωρίς κάποια οργάνωση.
Αυτή η συμπυκνωμένη πληροφορία, η οποία στα μάτια του χρήστη μπορεί να μοιάζει με διαφήμιση, αποτελεί εμπόδιο που δυσκόλευει τους χρήστες να βρούν το περιεχόμενο που αναζητούν. Ένα παράδειγμα αντίθετο της απλότητας αποτελούν τα κίνουμενα κείμενα και εικόνες.

Ένας επισκέπτης ο οποίος αναζητά τις τελευταίες ενημερώσεις στον εργασιακό τομέα και τις πιο πρόσφατες αλλαγές, δεν γνωρίζει που να εστιάσει στον ιστόχωρο, καθώς τα \textbf{Δελτία Τύπου}, που μπορεί να το ενδιαφέρει, κινείται συνεχώς και θα πρέπει να περιμένει να φτάσει στην αρχή, για να διαβάσει αν βρίσκεται εκεί η πληροφορία που τον ενδιαφέρει. 

\subsection{Οι χρήστες είναι ανυπόμονοι}
Οι χρήστες δεν αναζητούν τον πιο γρήγορο τρόπο για να βρούν την πληροφορία που τους ενδιαφέρει και δεν κοιτούν την ιστοσελίδα με διαδοχικό τρόπο πηγαίνοντας απο τη μια σελίδα στην επόμενη. Αντιθέτως, διαλέγουν την πρώτη σχετική (με αυτό που αναζητούν) επιλογή και μόλις βρούν τον σύνδεσμο που φαίνεται πως τους ικανοποιεί, τον επιλέγουν. Τέτοιες κατευθυντήριες επιλογές απουσιάζουν. 

Για παράδειγμα δεν υπάρχει ξεκάθαρο \textlatin{section} για ενημερώσεις και οδηγίες σχετικά με τις αλλαγές στον εργασιακό τομέα λόγω του \textlatin{Covid-19}. Παράλληλα αυτό που υπάρχει είναι ένα μικρό κινούμενο πλαίσιο, το οποίο χάνεται μέσα στο χάος των πληροφοριών και δεν κερδίζει έυκολα την προσοχή του επισκέπτη.

\subsection{Ποιός είμαι}
Με την είσοδο του επισκέπτη στον ιστόχωρο δεν υπάρχει κόποιο κουμπί ή κάποιος τρόπος να  προσανατολιστεί ο ενδιαφερόμενος ανάλογα με το ρόλο που έχει (εργοδότης, εργαζόμενος, άνεργος ή συνταξιούχος) για το που πρέπει να πάει. Αυτό αυτομάτως δημιουργεί μια σύγχηση στον χρήστη.

\subsection{Σχεδιασμός}
Όταν ο χρήστης επισκέπτεται για πρώτη φορά μια ιστοσελίδα, η πρώτη ενέργεια που κάνει είναι να κοιτάξει το σχεδιασμό και να χωρίσει τα εύκολα μέρη που περιέχουν τις  πληροφορίες που ψάχνει. Η περίπλοκη δομή μια σελίδας είναι πιο δύσκολο να διαβαστεί. Στον συγκεκριμένο ιστοχώρο για τον διαχωρισμό δύο τμημάτων έχει προτιμηθεί μια λευκή γραμμή, γεγονός που δεν παρέχει στον χρήστη την αίσθηση της οπτικής ιεραρχίας και καθιστά πιο δύσκολο για εκείνον την αντίληψη του περιεχομένου της ιστοσελίδας. Επιπρόσθετα, ενώ όλα τα τμήματα είναι σχεδόν κολλητά δίνεται η αίσθηση οτί υπάρχει επιπλέον πληροφορία δεξιά και αριστερά, στην πραγματικότητα είναι κενός χώρος. Τέλος, όταν μικραίνει το παράθυρο η σελίδα δεν προσαρμόζεται σε αυτό, στοιχείο που δυσκλεύει την πλοήγηση του χρήστη. 

\section{Ερώτημα 2o}

\subsection{}
Όταν κάποιος χρήστης αναζητά πληροφορίες σχετικές με τον \textlatin{Covid-19}, το πρώτο πράγμα που θα πατήσει είναι το σχετικό μπλε πλαίσιο. Τότε θα ανοίξει σε άλλη σελίδα ένα \textlatin{pdf} με ορισμένες οδηγίες.  Αρχές του Nielsen που 
\newpage
\bibliographystyle{plain}
\bibliography{biblist}

\end{document}